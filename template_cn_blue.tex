%# -*- coding:utf-8 -*-
%% start of file `template_en.tex'.
%% Copyright 2006-1008 Xavier Danaux (xdanaux@gmail.com).
%
% This work may be distributed and/or modified under the
% conditions of the LaTeX Project Public License version 1.3c,
% available at http://www.latex-project.org/lppl/.


\documentclass[11pt,a4paper]{moderncv}

\usepackage{fontspec,xunicode}
\usepackage[slantfont,boldfont]{xeCJK}
\usepackage{xcolor}  % replace by the encoding you are using


% \setmainfont{Tahoma}
\setmainfont{Times New Roman}  % 缺省英文字体.serif是有衬线字体sans serif无衬线字体
\setCJKmainfont[ItalicFont={Kai}, BoldFont={Hei}]{STSong}  % 衬线字体 缺省中文字体为
\setCJKsansfont{STSong}
\setCJKmonofont{STFangsong}  % 中文等宽字体
%-----------------------xeCJK下设置中文字体------------------------------%
\setCJKfamilyfont{song}{SimSun}  % 宋体 song
\newcommand{\song}{\CJKfamily{song}}
\setCJKfamilyfont{fs}{FangSong_GB2312}  % 仿宋2312 fs
\newcommand{\fs}{\CJKfamily{fs}}
\setCJKfamilyfont{yh}{Microsoft YaHei}  % 微软雅黑 yh
\newcommand{\yh}{\CJKfamily{yh}}
\setCJKfamilyfont{hei}{SimHei}  % 黑体  hei
\newcommand{\hei}{\CJKfamily{hei}}
\setCJKfamilyfont{hwxh}{STXihei}  % 华文细黑  hwxh
\newcommand{\hwxh}{\CJKfamily{hwxh}}
\setCJKfamilyfont{asong}{Adobe Song Std}  % Adobe 宋体  asong
\newcommand{\asong}{\CJKfamily{asong}}
\setCJKfamilyfont{ahei}{Adobe Heiti Std}  % Adobe 黑体  ahei
\newcommand{\ahei}{\CJKfamily{ahei}}
\setCJKfamilyfont{akai}{Adobe Kaiti Std}  % Adobe 楷体  akai
\newcommand{\akai}{\CJKfamily{akai}}


%------------------------------设置字体大小------------------------%
\newcommand{\chuhao}{\fontsize{42pt}{\baselineskip}\selectfont}  % 初号
\newcommand{\xiaochuhao}{\fontsize{36pt}{\baselineskip}\selectfont}  % 小初号
\newcommand{\yihao}{\fontsize{28pt}{\baselineskip}\selectfont}  % 一号
\newcommand{\erhao}{\fontsize{21pt}{\baselineskip}\selectfont}  % 二号
\newcommand{\xiaoerhao}{\fontsize{18pt}{\baselineskip}\selectfont}  % 小二号
\newcommand{\sanhao}{\fontsize{15.75pt}{\baselineskip}\selectfont}  % 三号
\newcommand{\sihao}{\fontsize{14pt}{\baselineskip}\selectfont}  % 四号
\newcommand{\xiaosihao}{\fontsize{12pt}{\baselineskip}\selectfont}  % 小四号
\newcommand{\wuhao}{\fontsize{10.5pt}{\baselineskip}\selectfont}  % 五号
\newcommand{\subwuhao}{\fontsize{10pt}{\baselineskip}\selectfont}  % 次五号
\newcommand{\xiaowuhao}{\fontsize{9pt}{\baselineskip}\selectfont}  % 小五号
\newcommand{\liuhao}{\fontsize{7.875pt}{\baselineskip}\selectfont}  % 六号
\newcommand{\qihao}{\fontsize{5.25pt}{\baselineskip}\selectfont}  % 七号


% \usepackage{fontawesome}
% \setCJKmainfont[BoldFont={WenQuanYi Micro Hei/Bold}]{WenQuanYi Micro Hei}
% \defaultfontfeatures{Mapping=tex-text}
% \XeTeXlinebreaklocale "zh"
% \XeTeXlinebreakskip = 0pt plus 1pt minus 0.1pt
% moderncv themes
\moderncvtheme[blue]{classic}  % optional argument are 'blue' (default), 'orange', 'red', 'green', 'grey' and 'roman' (for roman fonts, instead of sans serif fonts)
% \moderncvtheme[green]{classic}  % idem
% \moderncvtheme[blue,roman]{hht}
% character encoding



% adjust the page margins
\usepackage[scale=0.9]{geometry}
% \setlength{\hintscolumnwidth}{3cm}  % if you want to change the width of the column with the dates
% \AtBeginDocument{\setlength{\maketitlenamewidth}{6cm}}  % only for the classic theme, if you want to change the width of your name placeholder (to leave more space for your address details
\AtBeginDocument{\recomputelengths}  % required when changes are made to page layout lengths

% personal data
\firstname{凯丰}
\familyname{严}
\title{KaifengYan}  % optional, remove the line if not wanted
% \address{东营}{}  % optional, remove the line if not wanted
\address{2003/02/26}{}  % optional, remove the line if not wanted
\mobile{18654662297}  % optional, remove the line if not wanted
% \fax{fax (optional)}  % optional, remove the line if not wanted
\email{yankf21@163.com}  % optional, remove the line if not wanted
%\homepage{Blog: http://geekplux.com}  % optional, remove the line if not wanted
\social[github]{https://github.com/icn1Vad}
\extrainfo{%
  %LinkedIn: https://cn.linkedin.com/in/xxx \\
  WeChat: 18654662297 \\
  QQ: 2684435254
}

\photo[64pt]{avatar.png}  % '64pt' is the height the picture must be resized to and 'picture' is the name of the picture file; optional, remove the line if not wanted
% \quote{China\TeX 您的LaTeX乐园,TeX\&\LaTeX 王国}  % optional, remove the line if not wante

% \nopagenumbers{}  % uncomment to suppress automatic page numbering for CVs longer than one page


%----------------------------------------------------------------------------------
%            content
%----------------------------------------------------------------------------------
\begin{document}
\maketitle
\vspace*{-14mm}

\section{工作经历}
\cventry{23.07-23.09}{数据分析师}{中石化胜利油田物探研究院信息技术研究室}{http://bukegaoren.com}{}{完成物探AI模块集成项目,参与胜利信息网的运维和后台功能维护工作}
%\cventry{15.01-15.12}{不知道叫什么的公司名称}{不可告人的项
  目}{http://bukegaoren.com}{}{独立编写了根本编不下去的项目简介,可能还是凑字数
  比较好,来凑字数吧来凑字数吧来凑字数吧。出色的完成了凑字数的工作,并获得了最佳
  凑字数员工奖}
\section{教育经历}
\cventry{21.09-25.07}{本科}{兰州大学}{计算机科学与技术}{专业成绩:GPA 3.31/5(专业前30%)} % arguments 3 to 6 are optional
\cvlistitem{全国大学生电子商务“创新、创意及创业”挑战赛省级三等奖、校级二等奖}
\cvlistitem{第六次、第七次全国心理剧大赛三等奖}
%\cventry{15.09-18.06}{硕士}{和尚庙大学}{软件工程}{实验室 XXX 导师 XXX}{主要研究了
  人工智能,图形学,编译原理,机械键盘的拆装,快递包装的暴力拆解,颈椎与视觉保
  养,抹平小腹,治疗腰椎间盘突出}  % arguments 3 to 6 are optional
\section{项目}
\subsection{科研项目}
\cvline{药品通}{作为成员参与开发根据患者病况推荐药品的app,使用vue框架搭建应用端,并通过爬虫爬取相关药品信息}
\cvline{GPT-Mindmap}{作为成员搭建可根据用户话题或要求生成递归或非递归思维导图的网站,通过Flask框架搭建网站,并连接后端数据库、Xmind软件接口和ChatGPT接口}
\cvline{对twitter上印尼车祸信息的核实分析预测}{担任负责人,使用pyspark处理大数据集,使用word2vec和GloVe处理数据并可视化,建立相关模型,同时训练微调LLaMA模型}
\cvline{智能AI坐姿仪——正姿宝}{23.02-23.07 \\
  \textbf{职责}:负责视觉算法设计,基于TensorFlow.js的MoveNet实现人体17个关键点定位,通过监督学习对坐姿端正/不端正图片进行分类标注,并训练神经网络模型。 \\
%\subsection{开源项目}


\section{技能}
\cvline{\textbf{前端}}{熟练掌握Vue框架,具备前端开发能力}
\cvline{\textbf{后端}}{熟练使用Flask框架进行后端开发,熟悉数据库连接与操作}
\cvline{\textbf{数据}}{熟练运用Python进行数据处理以及爬虫,掌握pyspark处理大数据集,熟悉word2vec、GloVe等数据处理技术,了解数据库操作}
\cvline{\textbf{深度学习}}{熟悉常用的TensorFlow、PyTorch深度学习框架,掌握机器学习/数据挖掘、nlp、cv相关模型}
\cvline{\textbf{云计算}}{熟练掌握Shell编写技能,熟悉Kubernetes和容器相关技术,了解云计算架构体系和Hadoop生态系统,包括HDFS、MapReduce、Hive、HBase等,掌握Spark流处理框架}
\cvline{\textbf{其他}}{动手能力强,有一定3D建模能力以及硬件知识}



%\section{Publications}
%\cvline{已录用}{张三,李四,王麻子 基于 latex 的简历凑字数研究[C]// CVChina. 2025.}

% \subsection{Vocational}
% \cventry{year--year}{Job title}{Employer}{City}{}{Description}  % arguments 3 to 6 are optional
% \cventry{year--year}{Job title}{Employer}{City}{}{Description}  % arguments 3 to 6 are optional
% \subsection{Miscellaneous}
% \cventry{year--year}{Job title}{Employer}{City}{}{Description line 1\newline{}Description line 2}% arguments 3 to 6 are optional

% \section{Languages}
% \cvlanguage{language 1}{Skill level}{Comment}
% \cvlanguage{language 2}{Skill level}{Comment}
% \cvlanguage{language 3}{Skill level}{Comment}

% \section{Computer skills}
% \cvcomputer{category 1}{XXX, YYY, ZZZ}{category 4}{XXX, YYY, ZZZ}
% \cvcomputer{category 2}{XXX, YYY, ZZZ}{category 5}{XXX, YYY, ZZZ}
% \cvcomputer{category 3}{XXX, YYY, ZZZ}{category 6}{XXX, YYY, ZZZ}

% \section{Interests}
% \cvline{篮球}{\small 体力与技巧}
% \cvline{hobby 2}{\small Description}
% \cvline{hobby 3}{\small Description}

% \renewcommand{\listitemsymbol}{-}  % change the symbol for lists

% \section{Extra 1}
% \cvlistitem{Item 1}
% \cvlistitem{Item 2}
% \cvlistitem[+]{Item 3}  % optional other symbol% XeLaTeX can use any Mac OS X font. See the setromanfont command below.
% Input to XeLaTeX is full Unicode, so Unicode characters can be typed directly into the source.

% The next lines tell TeXShop to typeset with xelatex, and to open and save the source with Unicode encoding.

% !TEX TS-program = xelatex
% !TEX encoding = UTF-8 Unicode

% \section{Extra 2}
% \cvlistdoubleitem[\Neutral]{Item 1}{Item 4}
% \cvlistdoubleitem[\Neutral]{Item 2}{Item 5}
% \cvlistdoubleitem[\Neutral]{Item 3}{}

%% Publications from a BibTeX file
% \nocite{*}
% \bibliographystyle{plain}
% \bibliography{publications}  % 'publications' is the name of a BibTeX file

% \begin{thebibliography}{}
% \bibitem[]{} 移动增强现实可视化综述[C]. ChinaVis 2017.
% \end{thebibliography}


\end{document}


%% end of file `template_en.tex'.

%%% Local Variables:
%%% mode: latex
%%% TeX-command-extra-options: "-shell-escape"
%%% TeX-master: t
%%% TeX-engine: xetex
%%% End:
